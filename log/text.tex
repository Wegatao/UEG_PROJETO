backend/
│
├── app.py                # Arquivo principal do Flask
├── config.py             # Configurações do Flask e banco
├── models.py             # Modelos do banco de dados
├── routes/
│   ├── auth.py           # Cadastro e login
│   ├── peripherals.py    # Cadastro, listagem e atualização de periféricos
│   └── requests.py       # Solicitação de doações
└── database.db           # Banco SQLite


1️⃣ Tabela users (usuários)

Guarda informações de quem está cadastrando e recebendo doações.

Campo	Tipo	Observação
id	INT	Chave primária, auto-increment
name	VARCHAR(100)	Nome do usuário
email	VARCHAR(100)	Único, para login/cadastro
password	VARCHAR(255)	Senha criptografada
created_at	DATETIME	Data de criação
2️⃣ Tabela peripherals (periféricos)

Guarda informações dos periféricos que estão sendo doados.

Campo	Tipo	Observação
id	INT	Chave primária, auto-increment
user_id	INT	Relaciona ao usuário que está doando
name	VARCHAR(100)	Nome do periférico (Mouse, Teclado…)
description	TEXT	Detalhes do item
condition	VARCHAR(50)	Estado do item (novo, usado…)
created_at	DATETIME	Data de cadastro
3️⃣ Tabela requests (solicitação de doação)

Guarda quem solicitou determinado periférico.

Campo	Tipo	Observação
id	INT	Chave primária, auto-increment
peripheral_id	INT	Relaciona ao periférico solicitado
requester_id	INT	Usuário que solicitou
status	VARCHAR(50)	Status da solicitação (pendente, ok)
created_at	DATETIME	Data da solicitação

💡 Relacionamentos:

users → peripherals → 1:N (um usuário pode doar vários periféricos)

users → requests → 1:N (um usuário pode solicitar vários itens)

peripherals → requests → 1:N (um periférico pode ter várias solicitações)